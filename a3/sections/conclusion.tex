\section{Conclusion}
\label{sec:conclusion}
This section is about the observations and conclusion that were made while working on and with Deep Dream.

% Test: Image modifizieren mit dem Ziel es bestimmt Klassifizieren zu lassen, anschließend diese Behauptung prüfen
Generally speaking one can see that lower layers tend to select more abstract features such as lines, curves or waves building textures, while higher layers often select more concrete features such as eyes, body parts of animals or vegetation. Examples for this were presented in section \ref{sec:evaluation}.

As described in section \ref{sec:repeating-features}, some learned features seem to repeat across layers in a rotated way.
This observation supports a claim found in the keras blog.\cite{keras-blog}
They found that convolutional layers are not rotation invariant.
If one were able to find a method to make them rotation invariant, it could reduce the required layer amount by a large factor.


\section{Further Work}
With Deep Dream it is also possible to create entire videos.
This can be achieved through zooming.
Starting with a single dream, one applies a little zoom and dreams on that image. This process can be repeated indefinitely.
The resulting images can then be as video frames.
One of many examples for this is a video created by the user Johann Nordberg.\cite{https://vimeo.com/132700334}
We choose to present this examples, because one can see the different levels of abstraction throughout the layers.
This video starts with lower and works its way up to the higher layers.


% TODO: Fool networks in previous oder future work