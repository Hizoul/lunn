\documentclass{article}[]
\usepackage[textwidth=15cm]{geometry}
\usepackage[table,xcdraw]{xcolor}
\usepackage{graphicx}
\usepackage{listings}
\usepackage{hyperref}
\usepackage{pdfpages}
\usepackage{csvsimple}
\usepackage{float}
\usepackage{csquotes}
\makeatletter
\newcommand\urlfootnote@[1]{\footnote{\url@{#1}}}
\DeclareRobustCommand{\urlfootnote}{\hyper@normalise\urlfootnote@}
\makeatother

\begin{document}
	\title{Neural Networks Assignment 3 - Google Deep Dream}
	\author{Anonymous}
	\maketitle
	\lstset{
		basicstyle=\ttfamily,
		keywordstyle=\bfseries,
		language=Java,
		frame=single,
		aboveskip=11pt,
		belowskip=11pt,
		breaklines=true,
		breakatwhitespace=false,
		showspaces=false,
		showstringspaces=false,
		numbers=left,
		stepnumber=1,    
		firstnumber=1,
		numberfirstline=true
	}
	
	\section{Abstract}
	
	\section{Introduction}
	% was machen wir eigentlich
	% https://research.googleblog.com/2015/06/inceptionism-going-deeper-into-neural.html
	
	\section{Problem Statement \& Motivation}
	% Warum machen wir das überhaupt? -> NN sind Blackboxes, haben zwar definierte Struktur, aber Zusammenspiel der Weights unbekannt
	% Und natürlich Spaß
	% https://research.googleblog.com/2015/06/inceptionism-going-deeper-into-neural.html
	
	\section{Related \& Previous Work}
	% Was hat Google und andere gemacht und was probieren wir
	\section{Display Convolutional Filters}
	\section{When it does start to begin to be a dream}
	
	
	\subsection{Google Deep Dream}
	\subsection{Yahoo Deep Dream}
	
	\section{How Does Dreaming Work}
	
	\subsection{General}
	% Einzelne Neuronen identifizieren und deren Features visualisieren, durch Verändern des Input Images, so dass die Neuronen mehr aktiviert werden
	% Niedrige Layer enthalten eher abstrakte Features we Linien, Kurven und Ecken während höhere Layer konkrete Features wie z.B. Augen oder Säulen abbilden
	\section{Guided Dreaming}
	% mit Reference Image
	
	
	
	\section{Data And Used Models}
	% vortrainierte Neural Networks erklären/erwähnen
	% alle basieren auf der Image Database "ImageNet"
	
	\section{Experiments And Results}
	\subsection{Without Reference Image}
	\subsection{With Reference Image}
	
	\section{Conclusion}
	% Test: Image modifizieren mit dem Ziel es bestimmt Klassifizieren zu lassen, anschließend diese Behauptung prüfen
	
	
	
	
	
	
	
	
	\bibliography{report} 
	\bibliographystyle{ieeetr}
	
	
\end{document}