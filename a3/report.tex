\documentclass{article}[]
\usepackage[textwidth=15cm]{geometry}
\usepackage[table,xcdraw]{xcolor}
\usepackage[hyphens]{url}
\usepackage{graphicx}
\usepackage{listings}
\usepackage[hidelinks]{hyperref}
\usepackage{pdfpages}
\usepackage{csvsimple}
\usepackage{float}
\usepackage{csquotes}
\makeatletter
\newcommand\urlfootnote@[1]{\footnote{\url@{#1}}}
\DeclareRobustCommand{\urlfootnote}{\hyper@normalise\urlfootnote@}
\makeatother

\begin{document}
	\title{Neural Networks Assignment 3 - Google Deep Dream}
	\author{Anonymous}
	\maketitle
	\lstset{
		basicstyle=\ttfamily,
		keywordstyle=\bfseries,
		language=Java,
		frame=single,
		aboveskip=11pt,
		belowskip=11pt,
		breaklines=true,
		breakatwhitespace=false,
		showspaces=false,
		showstringspaces=false,
		numbers=left,
		stepnumber=1,    
		firstnumber=1,
		numberfirstline=true
	}
	
\section*{Abstract}
Neural networks pose a big black box that limits development to trial and error in a certain degree.
\emph{Deep Dream} offers one possibility to look into convolutional neural networks.
Moreover it offers unique visually pleasant results.

By using the \emph{Caffe} framework we were able to visualize learned features of the famous \emph{Residual Network} developed by Microsoft Research.
In the process of dreaming a multitude of features ranging from abstract textures to high level objects were discovered.
To granularly control the features drawn, one can create a \emph{guided dream}, where features of a reference image are replicated onto another.
Furthermore applying a \emph{bilateral} filter can also change the features drawn by a dream.


\section{Introduction}
% was machen wir eigentlich
% https://research.googleblog.com/2015/06/inceptionism-going-deeper-into-neural.html


\section{Related \& Previous Work}
\label{sec:previouswork}
% Was hat Google und andere gemacht und was probieren wir
\subsection{Display Convolutional Filters}
% https://blog.keras.io/how-convolutional-neural-networks-see-the-world.html
\subsection{Google Deep Dream}
% differenzieren zwischen erstem topic und "dream"
% erläuterung deep dream


% https://arxiv.org/pdf/1312.6034v2.pdf - Deep Inside Convolutional Networks: Visualising Image Classification Models and Saliency Maps

% Gradient: Technik um größten Anstieg zu finden -> man nimmt also beim Trainieren den negativen Gradient: -Delta(->W) = [0.2 -1 ...] // GIbt an wie die Weights verändert werden sollten
% Numerische Approximation des Gradients: https://en.wikipedia.org/wiki/Numerical_differentiation
% Gradient Descent/ Full Backpropagation): mache es für alle Trainingsamples und nehme den durchschnitt
% Stochasitic Gradient Descent: Nehm ein paar und nehme den Durchschnitt
% Backpropagation: Determine how one trainign example want to change the weights and biases

\section{How Does Dreaming Work}
\label{sec:how}

This section will focus on the main techniques how dreaming actual work.
Similar to the training process of a network, the dreaming algorithm is an optimization problem.
But rather than training a network to classify samples, one take a pretrained network and modify the input that some neurons in the net are getting more activated.


\subsection{General}
When looking at the training process of a (Convolutional) Neural Network, one can see that the weights and biases are changed in such a way that a given error function gets minimized.
Because the error function, for instance \emph{mean squared error}, has a lot of dependencies to the weights and biases the derivation is very complex if not even infeasible.
That's why the \emph{Gradient descent} algorithm is used to find the (global) minimum\footnote{however, using the gradient descent often leads only to a local minimum} of the function by taking small steps into the direction of the minima.
The partial derivative $\frac{\partial C}{\partial w}$ in respects to the weights	$\frac{\partial C}{\partial b}$ and to the biases of the cost function C is used to calculate how quickly the cost changes when the weights and biases are changed.

So in order to learn a network there are two essential steps:
First there is the \emph{forward} pass, where a sample gets classified by the net.
After that the error made by the network is calculated with the cost function.
The second step, the \emph{backward} pass, consists of the algorithm of the Gradient Descent, often called \emph{backpropagation} in the context of Neural Networks.
Based on result of the error function, also called \emph{loss}, the gradients for every weight and bias is calculated.

The gradients can be interpret as the direction in which the function should step in order to maximize the loss.
Hence we want to minimize the loss we take the negative gradient and add them to the weights and biases.
To prevent too slow learning or \enquote{overshooting} a local minima a \emph{learning rate} is introduced which gets multiplied with the gradients.

%TODO: Vielleicht Layer als l_end beschreiben, damit man dann sowas sagen kann wie: Gradients von Layer 0 - l_end sind berechnet worden
As already said in the introduction of this section, the goal of the Deep Dream algorithm is not to minimize the cost function, but to modify the input to maximize the activations of certain neurons.
Concrete speaking one can choose a whole layer within the net or just some neurons of a layer which output should be maximized according to the L2 norm.
By a normal forward pass to the specified layer the activation values gets computed.
In order to start the backward pass now, the gradient of the specified layer has to be set.
This can be done in several ways, at this point just assume that the gradients are set to the same values as the
activation values of that layer.
By starting the backward pass the backpropagation algorithm composes the gradient of each previous layer to compute the gradient of the whole submodel by automatic differentiation.\footnote{TODO Quelle: \url{http://caffe.berkeleyvision.org/tutorial/forward_backward.html}}
That's why the gradients (or at least the values) of the specified layer has to be set to something, because every layer needs the gradients from its previous layer in order to work.


When the backward pass is done, all gradients are computed.
Obviously the gradients consists of many as values as there are weights and biases.
But with Deep Dream only the weights of the input layer are important, because these can be added to the input image.
By adding the gradients, the activation values at the specified layer will increase or decrease and with that, respectively the features will become more or less present.

\subsection{Optimizations}
\label{sec:optimizations}
Same as the normal training process of the network, the dreaming is done by using multiple iteration.
By repeating the process over and over again, the features become more and more present, obviously only to an extend.

Another thing that is commonly used is the \emph{step size}, also called \emph{learning rate} in Neural Networks.
Instead of just adding the gradients as they are, one can higher or lower the effect by choosing a $step size < 1$ and $> 1$ respectively.

Besides these two rather simple techniques \emph{octaves} are used.
The goal is of this procedure is to scale the image after every dream step.
By increasing the size of the image in such a way, the network (hopefully) tend to find new features and enhances them as well, which finally leads to even more spectacular pattern.
The algorithm starts with the smallest version of the image and do the normal dreaming process.
By calculating the difference between the original and the dreamed image the added or removed features are extracted.
For the next octave both the base image and the features of the previous octave get scaled by a scaling factor and are added up together.
With this technique it is possible to create very large images, which will contain more and more features as the process proceeds.

 
Another small tweak is to use a jitter, that shifts the image along the x and y-axis in a random interval before the forward and backward pass.
This will lead to different activation functions and hence to different gradients and finally to a different change to the image.
This jitter shift is undone, after the image is modified.
This small trick not just lead to different results, it also provides smooth transitions between remarkable enhanced features.

Because enhanced features can look harsh, especially with a high step size, a good way to smooth things out is to use a filter. The filter can either be applied after every iteration or after every octave.
In Section \ref{sec:todo} the best results are presented and the used filters mentioned.


\subsection{Guided Dreaming}
\label{guided-dreaming}
Instead of increasing the occurrence of present features within an image one can also use a reference image in order to achieve \emph{Guided Dreaming}.
The approach is to do a normal forward pass with a reference image and to save the activations at the specified layer.
During the dreaming these activations are set to the gradients at the specified layers.
Doing this leads to results, where the features of the reference image are enhanced in the input image.
If one would give a reference image of cats to the network, the Deep Dream algorithm will create/enhance the learned features of cats in the input image.

There are also different approaches of the guided dreaming.
For instance instead of just modify the input image according to the activations of the reference image, one could also select the best matching correspondences from the guide and the base image.
To achieve this, the dot-products between the saved and the current activations of the image is calculated in order to find the best matching ones.
This is done because it doesn't really make sense to enhance features in the original image, that aren't there at all.  \footnote{https://github.com/google/deepdream/blob/master/dream.ipynb}




\section{Used Models and Images}
\label{sec:data}
% vortrainierte Neural Networks erklären/erwähnen
% alle basieren auf der Image Database "ImageNet"
For the upcoming experiments in section \ref{sec:evaluation} we need to decide which network model we use and what images are feasible as source or \emph{guide image}.
As classification network we chose to use \emph{ResNet} because it is the current state of the art providing the lowest error rate and hence performs best.\cite{cnnComparison}

By analyzing various existing \emph{dreamed images}, we came to the conclusion that images with a lot of varying objects offering different features might lead to interesting results.
Hence we were looking for images containing: Landscape with varying vegetation, Houses, water because it offers a lot of noise to act upon, clouds because of the real-life relatability mentioned in section \ref{sec:visual-aspects}.
To find images usable for this purpose we took to \emph{Flickr} because it allows to search for images licensed under the \emph{creative commons}.
There we found three images of interest. Figure \ref{fig:imgbeestemarkt} is an image of the Beestemarkt in Leiden and offers three of the wanted features: water, clouds and buildings.
In addition to that a few people are also in the picture that might turn into interesting things.
The second image visible in figure \ref{fig:imgfield} is a closeup of some vegetation on a field offering a lot of small distinctive features. It contains two of the features we are looking for namely a landscape with varying vegetation and clouds. It is also advantageous that the grass land in the background is made up of different colors ranging from green to yellow to almost brownish.
The third image, portrayed in figure \ref{fig:imglandscape} is similar to the second one but offers different wanted features because it also includes small houses, a properly visible start of a forest, and also there are not many distinctive clouds but just one big grey shadow across the sky.


\begin{figure}[H]
	\minipage{0.32\textwidth}
	\centering
	\includegraphics[width=0.5\linewidth]{img/beestemarkt.jpg}
	\caption{Source image displaying the Beestemarkt in Leiden\cite{imgbeestemarkt}.}
	\label{fig:imgbeestemarkt}
	\endminipage\hfill
	\minipage{0.32\textwidth}
	\centering
\includegraphics[width=0.5\linewidth]{img/field.jpg}
\caption{Source image portraying a field with mountains in the background\cite{imgfield}.}
\label{fig:imgfield}
\endminipage\hfill
\minipage{0.32\textwidth}
\centering
\includegraphics[width=0.5\linewidth]{img/landscape.jpg}
\caption{Source image portraying a landscape with a small houses and trees\cite{imglandscape}.}
\label{fig:imglandscape}
\endminipage\hfill
\end{figure}

\section{Experiments And Results}
\label{sec:evaluation}
As described in section \ref{sec:visual-aspects} \emph{deep dreaming} works by selecting layers and neurons that should be activated more.
In order to select good layers for the dreaming, we let each layer be dreamt onto an image to handpick the layers that offer interesting features.
Impressive results of that can be seen in section \ref{sec:withoutguide}.

To get a more granular control of the intensified features, one can use a guide image.
A guided dream, described in section \ref{guided-dreaming}, will select neurons based on the guide image in that layer.
The results are presented in section \ref{sec:withguide}.


\subsection{Performance Considerations}
\label{sec:performance}
The runtime of the deep dream depends mostly on the size of the input image and the layer amount.
Using an image of $\approx$600x430 pixels, 30 layers, 10 iterations and 4 octaves can already take around 45 minutes to apply and uses up to 14GB of RAM\footnote{single threaded i7-5930k@3,5GHz, 32GB RAM}.
We also implemented the bilateral filter which will be applied after every step.\cite{bilateral}
This will also heavily slow down the runtime because the application of the filter takes a few seconds and is applied at every step of modification.


\subsection{Without Guide Image}
\label{sec:withoutguide}

As mentioned in section \ref{sec:data}, the analyzed model contains 50 layers, so we will not be showing the results for each one, but rather a selection that we deemed satisfactory.
Throughout these layers one can find all kinds of features varying from different colors, textures to small parts of objects.
Looking at (TODO: layer 15/18 + 13/14) one can see that the recognizable features within a layer are definitely not limited to a particular one and also includes transitions(layer 13/14).
These transitions can either distinguish pixelated regions to very sharp ones.
But it can also result in a transition from smooth curves to very straight lines (layer16/18).
Going further up in the layers we found one layer of particular interest, depicted in figure \ref{fig:layer-snake}.
It visualizes four distinct things at once which makes it unique.
There is a snake's head in the lower left part of the image as well as to the right of the very left house.
The field and large parts of the sky were transformed to look like snake scales.
Apart from these two already distinctive objects the center seems to contain the right part of a dog's face and above the sky looks more like dog fur than snake scales.  
If one looks for definite objects one can find eyes TODO: (layer 43), dogs(layer), 

Another interesting find is layer 12 depicted in figure \ref{fig:layer-artistic}.
If one ignores the lines and purple tint it looks like an artistic style filter was applied.
This supports DiPaolo's findings, mentioned in section \ref{sec:previous-work}, that by using the right combination \textit{deep dream} can be used to modify pictures such that they look like paintings.

Another applied change that can be observed in almost all layers is a slight purple to rainbowish tint that gets applied to the image, especially around the newly added features, as visible within the lines in figure \ref{fig:layer-artistic} and in the sky of figure \ref{fig:layer-snake}.

\begin{figure}[H]
	\centering
	\includegraphics[width=0.7\linewidth]{img/alpsted-landscape_res4f_branch2a.jpg}
	\caption{Layer 41 \emph{res4f\_branch2a}. Seems like snake scales with a snake head in the lower left.}
	\label{fig:layer-snake}
\end{figure}
\begin{figure}[H]
\centering
\includegraphics[width=0.7\linewidth]{img/alpsted-landscape_res3a_branch1.jpg}
\caption{Layer 12 \emph{res3a\_branch1}. Apart from the distinct lines it seems to apply an artistic filter.}
\label{fig:layer-artistic}
\end{figure}


%TODO: beispiel hohe stepsize => visualisierung von haus o.a.
\subsection{With Guide Image}
\label{sec:withguide}

\subsection{Repeating Features}
\label{sec:repeating-features}

Within the different layers we found some to behave quite similar.
Previously we presented layers that differ from each other in form, size and shape.
Here we present layers that seem to have learned almost the same features.
To show the behavior in the clearest way possible we applied dreaming onto the same random generated noise.
As one can see in the figures \ref{fig:rotated_feature_1} \ref{fig:rotated_feature_2}, the created images look almost the same, the only real difference is the orientation.
It looks the enhanced features were simply rotated.

\begin{figure}[H]
	\minipage{0.49\textwidth}
	\centering
	\includegraphics[width=1\linewidth]{img/rotated_feature_1.jpg}
	\caption{Dreamed onto random now with the \enquote{res2c\_branch2b} layer}
	\label{fig:rotated_feature_1}
	\endminipage\hfill
	\minipage{0.49\textwidth}
	\centering
	\includegraphics[width=1\linewidth]{img/rotated_feature_2.jpg}
	\caption{Dreamed onto random now with the \enquote{res2c\_branch2c} layer}
	\label{fig:rotated_feature_2}
	\endminipage\hfill
\end{figure}



\section{Conclusion}
% Test: Image modifizieren mit dem Ziel es bestimmt Klassifizieren zu lassen, anschließend diese Behauptung prüfen



\bibliography{report} 
\bibliographystyle{ieeetr}
	
	
\end{document}