\documentclass{article}
\usepackage{graphicx}
\usepackage{listings}
\usepackage{hyperref}
\usepackage{pdfpages}
\usepackage{float}
\makeatletter
\newcommand\urlfootnote@[1]{\footnote{\url@{#1}}}
\DeclareRobustCommand{\urlfootnote}{\hyper@normalise\urlfootnote@}
\makeatother

\begin{document}
\title{Neural Networks Assignment 1}
\author{Oliver Scherp\\Matthias Mueller-Brockhausen}
\maketitle
\lstset{
  basicstyle=\ttfamily,
  keywordstyle=\bfseries,
  language=Java,
  frame=single,
  aboveskip=11pt,
  belowskip=11pt,
  breaklines=true,
  breakatwhitespace=false,
  showspaces=false,
  showstringspaces=false,
  numbers=left,
  stepnumber=1,    
  firstnumber=1,
  numberfirstline=true
}
\section{Excercise 1}
This is the table for our cloud distances.
The lowest and hence most similar values we found were 5.43, 6.01, 6.12 and 6.4.
These values were for classifying 7 to 9, 4 to 9, 3 to 5 and 9 to 8.
This does make sense given that a 9 can be, if written by hand, sometimes hard to distinguish from a 7, 4 or 5.




\end{document}